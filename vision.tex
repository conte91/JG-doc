\section{Recognition and pose estimation pipeline} \label{sec:linemod-pipeline}

In this chapter a pipeline similar to the one introduced in
\cite{linemod-pipeline} is presented, which is used to detect and estimate
the pose of the objects to be recognized. First a template matching approach is
used to detect the objects, then matches are progressively excluded (false
positives detection) based on hue values; finally valid matches are
progressively refined by physically moving a precision depth camera and
performing depth-based alignment of the match and the real object.

\subsection{False positive rejection based on hue values}
The first step of pose matching is made by purpose to be overly tolerant in
template matching. This is because the used objects are quite small and simple
in shape - which can reduce the number of features to match - and at the first
step are seen from far away ( more than 1 meter ) in order to maximize the
field of view without increasing the needed number of cameras. The counterpart
for being sure to match \emph{at least} something is having a huge number of
  false positives, which must be recognized and excluded before starting the
  alignment.


To do this, after performing each match together with a rough pose estimation,
the object is rendered again into its estimated position, and the portion of
source image corresponding to the render is taken. The two images are then
compared on a per-pixel basis, considering only the pixels that lie on the
depth mask created during rendering. 

Before comparing, the two images are converted to the HSV colour space, and
comparation is done based on the hue values. This allows to only consider the
actual colour of the model and match, and discard all information about
surface behaviour with respect to light, together with variations in the images
depending on different lightning conditions.

HSV colour space has singularities on the hue dimension when the chosen pixel is
either black or white; in this case, in effect, the hue of the pixel is
undefined due to the low saturation or low luminosity (\emph{value}). To fix
this, white and black pixels are turned to be zero-saturated, zero-valued
pixels with fixed hue, corresponding to yellow (for white) and blue (for
black). When matching, these are treated as special values and two pixels are
said to match either if their actual hue match (which
is robust to moderate lightning changes) or if their modified hue, saturation
and values match
(which happes if both the model and the scene appear to be actually white or
black). 

After computing the new HSV values, the four resulting images (two belonging to
the source image, two belonging to the rendered one) are scaled down of a
factor $K_s$. This makes the four images to be downsampled, and improves the
success rate of the subsequent step as it does minimize small alignment errors
due to the pose estimation error within the template. The render's mask $m$ is
scaled down of the same factor.

After scaling, the mask is first filtered in order to keep only the points for
which the mask's value is 1 (thus correcting the interpolated values on the
border which come out of the scaling algorithm) and then
eroded by a small amount (1 pixel). This allows to further remove the noise
which usually comes into the object's borders.

Finally, for each pixel belonging to the newly created mask, a comparison if
done between the corresponding render's and scene's pixel. If their difference
is below a certain threshold $t$, the two pixels correspond. The corresponding
pixels are counted and the matching ratio $P_{match}$ is computed with respect to the
total number of pixels belonging to the scaled mask:

\begin{equation}
P_{match}=\frac{\text{count}\left[(u,v) : m_{u,v}=1 \wedge \left( \lvert
\text{Hue}(r_{u,v})-\text{Hue}(s_{u,v})\rvert < t \vee R_{u,v}=S_{u,v} \right)
\right]
}{\text{count}\left[ (u,v) : m_{u,v}=1 \right] }
\end{equation}

Here, $r$ indicates the scaled-down render, $s$ indicates the scaled-down scene
portion, while $R$ and $S$ indicate the corresponding images with modified hues.

The whole matching and filtering part is repeated until at least one match
passes the filtering step. If no valid matches are found during the first
iteration, the threshold level for the LineMOD algorithm is decreased; in this
way, objects with small, partial occlusions can be recognized. If the threshold
value steps below a certain threshold, the algorithm fails and no valid poses
are returned.

\subsection{Point cloud alignments and pose refinement}
The matches that passed the first false-positive rejection step are saved and
sorted based onto their matching percentage. At this point, matches are probably
valid, but only colour information has been considered. Also, a good pose
approximation has been obtained for each match, but coming from a template, it
is not precise enough to be used for grasping. To overcome these problems,
another step is required for each match, which is pose refinement through point
clouds' alignment. In this step, colour information is dropped, and only depth
information is used. 

The Microsoft Kinect RGBD camera provides good depth informations for the template-matching
requirements, but it is not suitable for precisely find the pose of an object
for grasping. This is due mainly to the limitations of its depth sensor, having
a nonlinear depth error of about $1\%$\footnote{TODO:link at MS device page}.
Although this is quite a standard error, it must be combined with the minimum
depth range of $80\unit{cm}$ for the sensor. This leads to an error of about
$1\unit{cm}$ on the depth's values. Also, the field of view of the (fixed)
kinect does not allow to precisely detect the object's pose, as the depth error
increases the more the obejct is not centered with respect to the sensor. As a
higher precision is needed for good 
object grasping, a dedicated, precision depth sensor (\emph{Structure Sensor by
Occipital})has been mounted onto the robot's arm.


This camera's specifications\footnote{http://structure.io/developers} are much better than the kinect's ones, although it
does not provide RGBD images but only depth informations -- hence the need for two
different cameras; it has a much lower minimum range of $40\unit{cm}$ and is suitable for
precision depth measurements, having a maximum error of about $1\%$, which comes
down to $0.15\%$ at $40\unit{cm}$ of distance. This means having an absolute
error of less than $0.2\unit{mm}$ when the camera is close enough to the object.

As the matches already provided a good estimation of the object's position, the
iterative closest point algorithm explained into sec. \ref{sec:icp} can be
applied. 

First, a frame can be taken
First, the point cloud %TODO
for the scene is obtained as explained in sec. \ref{sec:cloud-reconstruction}.
As the mask of the render has already been computed, it is exploited at this
point and only the part of the scene corresponding to the actual object is
transformed to a 3D point cloud.

During the process of 3D reconstruction a lot of invalid values can be
generated: these are the points for which the depth sensor failed to obtain a
valid representation
