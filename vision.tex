\chapter{Object detection and pose estimation}
In this chapter a pipeline similar to the one introduced in
\cite{linemod-pipeline} is presented, which is used to detect and estimate
the pose of the objects to be recognized. First a template matching approach is
used to detect the objects, then matches are progressively excluded (false
positives detection) based on hue values; finally valid matches are
progressively refined by physically moving a precision depth camera and
performing depth-based alignment of the match and the real object.

\section{Linemod template matching}
%TODO

\section{False positive rejection based on hue values}
The first step of pose matching is made by purpose to be overly tolerant in
template matching. This is because the used objects are quite small and simple
in shape - which can reduce the number of features to match - and at the first
step are seen from far away ( more than 1 meter ) in order to maximize the
field of view without increasing the needed number of cameras. The counterpart
for being sure to match \emph{at least} something is having a huge number of
  false positives, which must be recognized and excluded before starting the
  alignment.


