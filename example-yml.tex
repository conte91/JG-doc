\chapter{Example YAML file generated by the software} \label{app:yml}
YAML is, like JSON, a file format used to describe a set of key-value
properties in a tree-like structure. However, unline JSON, which is
optimized for universality and easiness of implementation, YAML is
optimized for human readability and easiness to be hand-written. For
this reason, it is the choice format of the OpenCV's IO system. As
this particular module of the OpenCV libraries is particularly
extendable and well-designed, fitting well the project requirements
and the C++ way of thinking more in general, it has been used as the
standard system for data serialization. This lets a programmer to both
simply use this system into its own project, having an established
API on its side, and to easily debug and fix when needed the different
structures used by all the components of the framework, without the
need for a steep learning curve. Here follows an example of
YAML file containing the model of a RGBD camera, which the
\pre{CameraModel} class of the project is able to read and write
without effort from the programmer. It can be seen how it contains
both the intrinsic and extrinsic parameters, plus information on the
depth range and the resolution.

\paragraph{}
\listing{yml_list.lst}
