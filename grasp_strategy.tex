\section{Grasping poses' filtering, best grasp computation, high-level
  grasping strategy}  \label{sec:best-grasp-computation}
In this section the algorithm for selecting a set of good grasp from the set of
possible ones is presented, together with the high-level strategy for
order management, which is a closely related task for how the
implementation is done. The rationale to the  approach which is
presented here is to make the main order-management software aware of
which items can be easily grasped, and which will be difficult or
dangerous to take. In this way, by starting from the most simple ones,
it is possible for the hardest objects to take to have their way
freed and become so more accessible by the robotic arm.

When a set of poses has been generated for each object, either automatically or
manually, the main problem with it is that it will be a very big set (assuming
10 samples on each direction for each plane of the object, for example, would
lead to 600 samples for a cuboid, plus manually-added poses). As intersecting
objects is the most time-consuming task for this part, is thus
important to filter the poses in a good way.

In the project's specification it has been stated that the list of
objects to be grasped will come from a text file, and will have a
predefined bin from which the grasp will have to be done; this section
refers to this particular situation, but as it will be seen it is
trivial to make this algorithm use different constraints in order to
suite anyone's needs.

The implemented strategy goes as follows: when pose computation is required for an object,  the whole set of
its poses is put into a list and is then sorted base on preference
scores.
In this way, important poses will be tried first, and the
algorithm will be able to stop itself if a good pose has been
generated for the object which does not collide with anything into the
scene.

Taking out one by one the possible poses from this list, the main
function to compute volume intersection is called. Keeping in mind all
the heuristic strategies stated in the previous sections, the
operations described in the next paragraphs are performed.

First, an empty scene (represented as a list of shapes) is created,
located into the origin, and the shape $B$ of the bin for which the grasp has to be computed
is appended into it. This object is cached to speed up the next
computations, as each shape is able to cache its own point cloud
internally and the bin shape's approximation will probably be quite computationally
long to compute, being it a compound too.

The scene list, which is the list of objects which are present into
the bin, together with their poses, is built, and the object which has
to be taken -- identified by its index into the scene list -- is
removed. This will lead to a scene representing only objects that
the gripper will need not to bump in. This new, scene object, is then
transformed using the inverse of the object's transform: this will
make the scene make use of the object's coordinate system, in which
all the gripper's poses are.

Possible grasping poses are popped one at a time from the poses'
list. A new compound, representing the gripper into its corresponding
pose, is built now. Then the algorithm for shapes intersections is
called for each object into the scene. The score function of
sec.~\ref{sec:grasp_score} is then computed on this result,
weighting in by means of its $\alpha$ parameter, which is used as the
already introduced movement coefficient in order to define which
objects can be bounced into without risk and which are hard or
impossible to touch; for example, the bin object will have a very high
$\alpha$ value, while lightweight objects will have $\alpha \approx
1$. At the end, a total score is computed by summing up all these
values, intersecting the gripper with the scene both into its approach and its
gripping pose.

If the total grasp score is approximately 0, i.e. below a threshold $\epsilon$, this means that the pose is
the best one, so the algorithm can stop. If the returned value is
greater than 0, a better pose could be still in the list, with a worse
preference value: in this case, the algorithm should prefer it
first. Thus, the algorithm moves on with finding another pose.

At the end of this search loop, the current best pose will be the
searched one both in terms of obstacles avoidance  and preference
score. 

\section{High-level order management} \label{sec:order-list}
With the grasping strategy described into
sec.~\ref{sec:best-grasp-computation}, it is possible to compute the
best possible grasp for each object, regardless of the current order
state, and assuming to be in an ideal situation in which the exact
pose is well-known for each object into the bin. In practice, this is
at the very least ingenuous: first, the recognition algorithm could
fail for some objects due to high occlusion or poor illumination
conditions. Second, some runtime errors (for example, the robot not
being sure whether a grasp succeeded in the past) could lead a bin to a
situation in which the actual state of the bin is totally
unknown. Last, no grasping poses could have been found for an object
which guarantee no intersections at all with the rest of the scene. In
this sense, it is bad to assume that, given the best grasps for every
object, the order in which these are executed does not influence the
success of the algorithm. Given the list of objects which have to be
taken, it is thus important to \emph{sort} them in a simple way, in
order to first fulfil simplest orders, and possibly making it easier
to fulfil the most difficult ones in the subsequent iterations.

\subsection{Order list' sorting strategy} \label{sec:sort-order}
Preference score is already used when computing single grasps as a field on which ordering of
grasps is performed: however, as explained, it is preferred to be able
force a grasp to move up or down as needed. Thus, a more complete strict
ordering for grasps have been implemented:
\begin{itemize}
\item {If a grasp has been manually flagged as good, it has
  precedence over every other grasp which has been not; this is
  useful, for example, if the object is known to be the only one into
  its bin;}
\item{If a grasp has been manually flagged as bad, every other object
  which has been not has precedence over it; this can be used to mark
  objects into a bin for which the state is unknown;}
\item{The total volume of every object which has not been found into
  the bin in which the grasp has to be done is summed: grasps with a
  lower value (which is indicative of lower uncertainty over
  obstacles) have precedence over grasps with a higher one. This
  will ensure that grasps are done first if objects have been
  correctly mapped into the bins' space;}
\item{In all other cases, the normal intersection score introduced
  in \ref{sec:grasp_score} is applied.}
\end{itemize}

With this final taxel, the robot is capable of full automatic
operation. In particular, given an order list as an input, this is the
serie of steps which it will be able to do in order to complete it at
the best of its possibilities:

\begin{enumerate}
\item{At initialization time, the gripper model $G$ is read from a
  model file and stored into memory; a set of bins' position is read
  from a file too; these are treated as a set of local reference
  frames on which shapes $B_i$ will be placed when searching for a
  grasp on a specific bin, in order to consider the bins' borders; the
  bin's shapes $B_i$ are read from file too.}
\item{The input JSON file is parsed, and the contents of each bin are
  updated with the set of contained objects. When searching for a
  particular object into a bin, every object's position will be actually
  identified into it. At this time, objects' positions are all set to
  local bin origin $(0,0)$ and every bin is marked as \emph{dirty}.}
\item{A table is built to store the list of objects for which a
  valid pose has already been computed: as the poses of objects won't
  change if the robotic arm will not pass near them (i.e. in their
  bin), this will make it possible to reuse already-computed poses for
  every bin until they are actually executed or become invalid;}
\item{The order list is scanned from the JSON file, and saved in
  memory;}
\item{All depth cameras are shut down to avoid mutual interference;}
\item{For each item into the order list, if the bin corresponding to
  the object to be grasped is marked as dirty, the positions of the
  objects into its local reference system are updated. To do this, the
  camera assigned to the object is turned on, a shot is taken, and the
  camera is turned off again. After this, the precision depth camera
  is moved in front of the bin and another shot is taken, again turning on
  and off the sensor. These images are saved into a global
  bin-to-frames table. For each object which is stored into the bin,
  the corresponding recognition algorithm (mainly, Line-MOD) is
  called; the recognition process will take care of optimizing this
  process, e.g. by calling a single instance of Line-MOD for all the
  objects which can be recognized this way; after a first pass, the
  same recognition algorithm will try to refine the pose of the known
  objects using the depth image obtained previously, as explained in
  detail into the whole sec.~\ref{sec:vision}; for all the objects for which
  these algorithms converge correctly, the corresponding count of
  properly detected objects is updated into the bin status and their
  pose is saved.}
\item{Each of the objects into the order list is checked to see
  whether it has been correctly recognized at the previous step; if it
did, and its bin is marked as dirty, the best grasping pose for it is
recomputed as described in sec.~\ref{sec:best-grasp-computation}; if
this object is the only one left into its bin, this pose is marked as good;}
\item{Each dirty bin is marked as not being dirty anymore;}
\item{The order list is sorted according to the strategy of
  sec.~\ref{sec:sort-order};}
\item{One order is popped from the top of the sorted order list, and
  the corresponding grasp is executed: first, the robotic arm moves in
  front of the bin (to have a safe starting position); then, it moves
  to the approach pose associated with the grasping pose, and finally
  to the  grasping pose itself. The gripper then returns back to the
  approach pose, and finally to the safe position;}
\item{The robot moves the object into the delivery bin and releases its grasp;}
\item{The bin from which the object has just been taken is marked as
  dirty;}
\item{If the order list is empty, the robot stops. Otherwise, the
  algorithm is restarted from point 5.}
\end{enumerate}

This is, obviously, a proof-of-concept algorithm: it does, however,
show how a robot can be programmed with extreme simplicity to
automatically grasp a complex set of heterogeneous items using the
framework developed into this project. Also, as every part of the
framework has been made with a way to expand and control it,
exploiting the strengths coming out from high-level C++ programming,
it can be seen how different strategies could be easily developed in
order to apply this same ideas with the use of different recognition
algorithms or gripping pose generations.
