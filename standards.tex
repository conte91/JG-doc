\section{Project-defined conventions}
If as a complete software solution the goal of this project is to
fulfill the requirements of the Amazon Picking Challenge, the
principal goal of the same as a software framework is to provide a
well-defined, expandable and easy-to-use base structure which can
compensate the fragmentation and lack of interoperability found in
most of the mainstream software libraries currently used in
research. The first important step towards providing the user with a
single, well-defined software interface to use for every task is the
definition of some standard practices, which have been used through
all of the code stack and are hereby presented:

\begin{itemize}
\item{Every module of the system uses the same unit system, which is
  composed of all the international system's measurement base units,
  without usage of multiples or submultiples; this is a strong
  requirement for a user, but is actually easy to implement without
  effort due to floating-point notations common to most programming
  languages, which permit at least the usage of scientific notation
  for floating-points values. External libraries, such as OpenNI RGBD
  camera libraries, which use different units are accessed by means of
  wrappers, which provide transparent interfaces to the user;}
\item{All the reference systems are right-handed, and every physical
  object or movable part is linked to a reference system; relative poses
  between objects into physical world are represented by means of
  relative transformations between their reference systems, and these in turn are represented by means
  of affine transformations, i.e. $4 \times 4$ transformation matrices
  formed by a linear (rotational) $3 \times 3$ part, a translation
  part, and a scaling factor;}
\item{All the matrices used into the project, including affine transformations, are represented 
  by each module using Eigen's matricial datatypes; Eigen's matrices
  provide a lot of simple ways to instantiate themselves starting from
  the most different datatypes (e.g. OpenCV's matrices, raw C
  pointers), which makes this choice good as a standard as it makes it
  extremely easy to build wrapper functions for every existing
  algorithm which doesn't follow this convention;}
\item{All the images are represented through the use of OpenCV
  matrices, and their pixels are represented by their $(u,v)$
  coordinates, having the $0,0$ coordinate into their top-left corners
  -- this means that, in practice, an image is treated as a matrix
  indexed by its row and column indices;}
\item{Data representation is done by means of YAML files (an example
  of which can be found in app.~\ref{app:yml}) using the OpenCV's
  APIs. These APIs make it easy to serialize and deserialize every
  object, in order to configure the initial status of a system or save
  it between different sessions, and are easily expandable in an
  object-oriented way: it is sufficient, for a new datatype which
  wants to be serialized or deserialized, to declare its own
  \pre{read} and \pre{write} functions which can write transparently
  YAML using C++'s standard \pre{ostream} syntax;}
\item{Inter-system portability of the project is guaranteed by the use
  of standard tools and languages; in particular, the C++11 language is
  the preferred one for system expansion, and no nonstandard extensions
  have been used. Using C++ as a core language makes it easy to write
  bidirectional bindings through the framework and other software: thus,
  additional modules can be written easily using a completely different
  language, but they shall come with C++ bindings at last in order to
  use them from every other module.
\end{itemize}

This small set of requirements has been found to be extremely
convenient to work onto, as it is easy to use, fits well the overall
structure of the code, and does not require significant effort from
the software developer.
