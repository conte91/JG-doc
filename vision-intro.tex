The main purpose of the first part of this project, which is explained
in detailed into this chapter, is the implementation and testing of a
flexible, easy-to-use, and scalable system for multiple rigid objects' recognition.

Although several methods have been proposed in the past, the research
in this field is still very active; a lot of algorithms have been
proposed in the past years, which can be distinguished in two
categories: in the first big group of object recognition algorithms,
either a set of salient points, called \emph{features}, of the image is
computed, or statistical data is collected over a set of ranges into
the image itself; then, the same operation is done for a model of the
object, whose representation is either taken online in a process
called \emph{training}, or given as an input into same standard
format. In the second one, a set of \emph{templates} is extracted for
each object which have to be recognized, which can be directly matched
into the source image: in practice, each template is considered as it
was a single object, and informations over the object which it refers
to is retrieved only after some have matched, by associating metadata
with each of them.

Although template-matching algorithms do generally provide a good
performance, in terms of recognition rate, they have found a hard way through
the last years, especially in the field of high-frame-rates systems,
mainly due to the relatively huge amount of data and training they require in
order to work properly; a recent algorithm, called Line-MOD, has been
proposed, which has the target to solve the main problems of speed and
training time common of the other ones, while providing the same high
recognition rate and low false-positive detection. Scope of this part
of the project is thus also to evaluate the performance of this
algorithm to find whether it is promising for a future application in contexts such as
the one presented in sec.~\ref{sec:apc}. In order to do this and in
the meantime keep the framework as modular and flexible as possible
for future use, the  whole object detection system has been developed
so that different recognition strategies can be assigned to each
object, leaving Line-MOD as a default algorithm suitable in a lot of
common cases.

During the rest of this chapter, first an overview of the software and
hardware tools used to solve the vision problem is done; then, the
developed recognition system is explained in detail, with particular
focus on how the OpenGL library works
(sec.~\ref{sec:opengl-rendering}) and how it has been exploited to
develop a system to programmaticaly train new objects and add them to
a database with no effort (sec.~\ref{sec:training}). Finally, the
Line-MOD algorithm used as a default for recognition is explained
(sec.~\ref{sec:linemod}), together with its actual implementation into
a full recognition pipeline (sec.~\ref{sec:linemod-pipeline}). The
latter is the main interface from which the libraries built relative
to this chapter can be called to perform no-effort recognition.
