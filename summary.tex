\documentclass[a4paper,11pt]{article}
\usepackage{adjustbox}
\begin{document}
\section*{A framework for automated object detection and gripping}
\paragraph{Candidate:} Simone Baratta
\paragraph{Supervisor:} Marco Ghirardi
\paragraph{Context:} Industrial automation
\paragraph{Keywords:} Robotics, Gripping, Computer Vision
\paragraph{}
Robotics has always been an appealing research field for many
application fields, starting with the radical technological changes
introduced by the industrial revolution, up to the most modern
robotic systems, which find their main application into factory
automatisation, especially in big factories' pipelines. However, all
of these approaches have lead mainly to strictly special-purpose
systems, which can thus not be applied to solve efficiently daily, general problems such as
robotic help for disabled people or research of missing people in
dangerous environments. In this sense, one of the most important
fields in which the state-of-the-art research is being pushed forward
the most is automated object grasping and manipulation, which would have
a variety of applications in heavily diversified fields.

As all the state-of-the-art research projects regarding this topic  mainly
consist of finite, demonstration projects, and most available
development frameworks for this kind of applications are either
outdated, nonflexible, or not maintained, this thesis project aims to
deliver a new, complete framework for automated object detection, pose
estimation and grasping, with good potentiality to be easily maintained for
future use in the robotics field. A complete detection and grasping solution is
built upon this, which can be extended to fit diverse needs.

\subsection*{Project objectives}
The case study which has been taken as a reference for this project is
the \emph{Amazon Picking Challenge}, a robotics competition which saw
its first edition if June, 2015, as part of the 2015 edition of the \emph{IEEE
International Conference on Robotics and Automation (ICRA)}. This
event has brought some of the world's top-level research 
centres to show the state of the art in this field. During the last
years, Amazon.com, Inc. has
added more and more automatic processes to their warehouses; in order
to complete the full automatisation of the company, this challenge
aimed to push researchers for a solution for the general object gripping
problem.

The objective of this project is to propose a full solution for the
problem statement of the Amazon Picking Challenge, building the
resulting grasping framework out of it. The implementation is based on a
COMAU's Racer999 robot, which is a small standalone robotic arm,
equipped with a multi-tool gripping end.

The requirements for the project have been:
\begin{itemize}
\item{From  the challenge specifications:
    \begin{itemize}
    \item{The robot must be able to take objects from a \emph{shelf},
      made of several \emph{bins}, whose global location and shape is known
      a-priori;}
    \item{The set of objects is known a-priori too; multiple objects
      of the same type can be present into the same bin;}
    \item{The robot must be \emph{totally independent} in its
      operation: input is given to it as a single, JSON-formatted
      file, and starting the grasping system is the only allowed
      human intervention into the whole process;}
    \item{Extreme care must be taken by the robot never to damage any
      object during manipulation, and never to take any object from a
      wrong bin or to take the wrong object. This is motivated by the
      project being applied in warehouse automation.}
    \end{itemize}
    }
\item{From the framework specifications:
  \begin{itemize}
  \item{The code structure must be extremely modular, in order to be
    easy to maintain, extend, and support;}
  \item{The framework must be totally portable to a variety of hardware
    platforms and software systems; for this, the C++11 programming language has been used for the
  whole as a standard, using only well-known, stable and portable libraries
(OpenCV, PCL, Eigen); this lets any future developer using this system to add
functionality to its own system at no cost and with the possibility of using
multiple programming languages at the same time (for example, python bindings have been generated
for many parts of the code);}
  \item{The framework must use a single, uniform standard convention through all
      of its components and data structures, in order to maximize the
    interoperability between this and other systems through simple wrappers.}
  \end{itemize}
  }
\end{itemize}

The whole source code of the framework has been made available through a
permissive, open-source license, in order to facilitate knowledge sharing in
this research area and be of help for future projects; it will be, together with
the grasping solution built upon it, used by the \emph{IsaacTeam} robotics research
project of Politecnico di Torino to participate to the 2016 edition of the
challenge.

\section*{Results}
The created framework fits the expectations: in particular, it contributes well
to the research fields it is put into because of the high modularity of the
whole system, which, as anticipated, has not been found in any of the
current available solutions; also, it provides the user with both a
working vision system and a working grasping strategy algorithm, which
are the main points in the solution of any grasping problem, included
the Amazon Picking Challenge one. Usage of this project in robotics environments
offers the user a single, well-defined interface for data interchange
between each of its components,
based on commonly accepted standards such as metric units for every
measure, or unique, well defined reference systems for each physical
object (camera, robot, objects) to be represented. Although
simple in concept, these standards of interoperability have been
lacking in most of the current robotics tools, mainly because of them
being composition of heavily heterogeneous libraries, each one
defining its own conventions for the specific application. This
framework can be instead used as a good starting point to build
homogeneous systems and applications, which can bring for sure
improvements to code safety, algorithms clearness, and more flexible solutions.

Each part of the system has been tested for full functionality before
starting to code the next one: this, and the usage of the C++
language, makes the created algorithms robust to use without modifications.

In particular, tests have been created for both the vision system and
the gripping one to evaluate their performance: as shown in
fig.~\ref{fig:test-vision}, the vision system is able to correctly
identify and estimate the pose of many rigid objects, even from a
quite high distance. The gripping system, in turn, is able to produce grasp
poses for each objects which minimize the probability of hitting other
objects during its movement, which has been a strong requirement for
the project as explained. Future extendability of the weaker
algorithms, such as the relatively slow training process, is
guaranteed by the possibility to swap components without undermining
the remaining ones, as in every object-oriented system; a good
possibility for doing this would be integration of each part of the
system with a binding to the ROS platform, which has now become a
standard environment in robotics research.

\begin{figure}[htbp] 
\centering
\begin{tabular}{c|c}
  \adjustbox{valign=m}{\includegraphics[width=2.2in]{./Results/recog}} &
  \adjustbox{valign=m}{\includegraphics[width=2.2in]{./Results/recog_out}}
\end{tabular}
\caption{The object detection system rendering (right) a searched object into
  the detected pose for a successful match on a cluttered scene (left).\label{fig:test-vision}}
\end{figure}
\end{document}
