\section{Offline pose training}
An algorithm have been produced which can automatically generate a set of
poses, or read a set of poses from file. In this way, poses for a given object
can be generated, and heuristic poses can be given by the programmer as a hint
for the grasping algorithm.  

Each grasping pose is associated with a \emph{preference score} $P_{g}$: this will be taken
into account as a second-level sorting value for grasps, meaning that the
grasping algorithm will first search for the grasp with the lower score (i.e.
intersection volume, as described in sec. \ref{sec:grasp_score}, but in case of
a tie, it will prefer the grasp with a higher preference score.

The parameter used to assign to each grasp a preference score is the supposed
easiness to grasp the object without having it dropped during the subsequent
movement phase: for example, for the duck-toy object shown in fig.
\ref{fig:duck-toy-grasps}, a grasp done with a sucker on the plastic covering
should work properly enough, but, being the plastic covering not rigid, the
object is not sure to be taken without possibility of falling. Instead, a grasp
done with clamps on the rigid, cardboard top of the envelope would work almost
for sure. In this case, both poses would be added to the grasp list, but the
latter would be preferred if both would not make the gripper collide with the
rest of the scene.

The pose generation has been splitted into two, minor problems: indentifying a
set of points onto the object which can be used for efficient grasping, and
assigning to each point a reference frame into the gripper's model; after these
informations have been combined, the desired gripping pose is well-defined, as the
whole pose of the gripper can be computed. Problem splitting is useful in this
case, as the two parts are independent and only a little number of gripper's
reference frames -- the one corresponding to the gripper's tools -- are actually
useful for grasping. On the other hand, the problem of  generating poses for a generic
reference frame can be solved by exploiting the geometric properties of the
object itself, and this makes the algorithm agnostic with respect to the actual
gripper's shape and thus more flexible and general.

The point of connection between the two subproblems has been constrainted at the
implementation's level for simplicity's sake: all of the generated pose
points will be associated to a vector at generation time. This vector will
coincide with the tool's $Z$ axis at the time of pose binding. When describing
the pose of the tools associated with a certain set of gripping poses, it will
be sufficient to set their $Z$ axis into the operative direction (e.g. suiction
direction for suckers) in order to have the generation algorithm automatically
bind points to an actual gripper's pose.

\subsection{Pose generations on a line and on a plane's surface}
If a set of poses on a plane surface must be generated, a good method is simply
to sample the plane onto equally distanced points. Starting from the center of
the plane, which would be the preferred point for the gripper to land onto,
samples can be taken at increasing distances until the border of the surface is
reached. when doing this, it assumed that the algorithm requiring pose
generation will already have taken into account a good margin from the gripper
to operate, and so none of the poses will actually result into a gripping
failure. With this in mind, a good preference score can be obtained as the
squared distance from the gripping point $\vec{p}$ to the center $\vec{c}$ of the surface:

\begin{equation}
P_{g}=(\vec{p}-\vec{c})(\vec{p}-vec{c})^\tau
\end{equation}

Regarding the pose's orientation, the $Z$ axis is well-defined, as it will
coincide with the plane's perpendicular vector. Two cases have been considered,
as it is useful to have two different descriptions for the plane's section: if
the plane is described using a width and a height vector, named $\vec{w}$ and $\vec{h}$ the perpendicular
direction is the direction of their product:

\begin{equation}
  \vec{z_{g}}=\vec{w} \times \vec{h}
\end{equation}

On the other hand, it is helpful sometimes to describe the plane by its
perpendicular direction and center point: in this case, the $Z$ axis will just
be the same as the perpendicular vector.



