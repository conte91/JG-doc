\section{Offline training of objects using 3D meshes and OpenGL}
In order to efficiently train the object database for each model, it is needed
to create a big set of templates, which is able to cover the whole set of view
points. Training can be both performed with either \emph{online} methods (i.e.
phisically looking at the object from different viewpoints with the chosen
camera) or \emph{offline} ones, which build the dataset starting from known
object informations such as pre-extracted colour features or shape informations.

In this section we present the proposed method for template database
creation, which is an implementation of the method used by modern template
matching algorithms such as \cite{TODO} and \cite{TODO}, and
has the advantage of being quite fast, fully automated for any dataset size,
and highly scalable (different dataset from different vendors can be merged
together or distributed with no effort at all).

\subsection{Offline training advantages}
It has been suggested in \cite{TODO} that fully-automated, offline training method can
outperform online training methods used as a de-facto standard
in previous years \cite{TODO}. Main advantages of offline training methods like
the proposed one, with respect
to online ones, include:

\begin{itemize}
\item{Offline training does not require human intervention. Human-based
training is error prone, and cannot yield to good results if not done by an
\emph{expert} person. In an industrial or research scenario, this implies
employing a skilled human resource in a non-productive task, which has obvious
consequences in terms of costs (at least if the process is repeated
for a big group of objects) and team productivity (the charged person will be
actually involved full-time in a task which is well below is actual
capabilities);
\item{Accurate training is a time-consuming task and physical objects must be available. In an
industrial scenario, this can mean a longer time-to-market or delayed custom products deli-
very. On the other hand, offline training does not require a phisical copy of the object: for
example, the proposed approach only needs a CAD model of the object, which in modern
scenarios is available well before the final product; this means that training can be done at
early project phases, before the objects used for testing the project actually
arrive;}
\item{If offline training is accurate enough, it can bring much more precise results than online
one. For example, pose information associated with each template are exact when generated
offline, while and subject to human error;}
\item{Large objects can be modelled without the need of physical movement around them. As
algorithms such as \cite{linemod-pipeline} will probably evolve their already good realtime capabilities, they will
be able to be applied to environments such as cities, in which the trackable objects are too
big to be modeled online at low cost.}

\subsection{CAD modeling of objects}
The idea behind the training procedure is to exploit CAD models of the object
to recognize in order to generate realistic renders of it and use the latter
for feature extraction and database training. In this way, a physical camera is
 simulated and full coverage which 
