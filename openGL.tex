\section{OpenGL rendering process} \label{sec:opengl-rendering}
In order to obtain good renderings for training and recognition purposes, it is
important to use the same model of camera described in sec.
\ref{sec:camera_modelling}.

In this chapter, this modelling is described. First, a brief
introduction on the correct usage of homogeneous coordinates (sec.
\ref{homogeneous-coordinates}) and OpenGL reference systems (sec.
\ref{opengl-reference-systems}) is done. After this, the OpenGL rendering
process is described (sec. \ref{opengl-rendering-process}), and finally the
implementation of objects' rendering for the purpose of this project is
described keeping all of the previous into account (sec. \ref{renderer3d}).

\subsection{Homogeneous coordinates} \label{homogeneous-coordinates}
Homogeneous coordinates were introduced by the famous mathematician August Ferdinand Möbius in
\cite{homogeneous-coordinates}; they are a common way to represent points in
\emph{projective geometry} by using an additional scale value, usually called
$w$, in represented
vectors, which can be seen as an inverted scale factor for the whole vector. In
practice this means that the association between homogeneous coordinates $H$ and
cartesian coordinates $C$ is:

\begin{equation}
H=\left(\begin{array}{c}wx\\wy\\wz\\w\end{array}\right) \Leftrightarrow
C=\left(\begin{array}{c}x\\y\\z\end{array}\right)
\end{equation}

Although this generates redundancy, as there will always be more freedom degrees
then constraints when mapping from coordinates to points (2D points are
represented with 3 values, 3D points with 4, and so on), this notation can
simplify a lot the computing effort for usual geometric operations -- for
example, scaling a set of points only requires to change their $w$ value.
Also, when performing matrix operations, this geometry has the good property to
be \emph{scale-invariant} for vectors, which means that coordinates are
multiplied by a constant value they keep representing the same point.

The last interesting property of homogeneous coordinates for the purposes of
this thesis is their capability to easily distinguish points from directions;
in fact, a directional vector can be represented as a point at the infinity,
and, differentely from cartesian coordinate systems, homogeneous coordinates
easily represent them as directional vectors by setting their $w$ coordinate to
$0$.

Also, \emph{all} affine transformations, if operating on homogeneous
coordinates, can be representing as $4\times 4$ matrices including rotation
matrix $R$, translation vector $T$, and scale properties $s$:

\begin{equation}
A=\begin{pmatrix}
R & T \\
0 & s
\end{pmatrix}
\end{equation}

By representing directional vectors as points at the infinity, they will have
the good property of never being scaled nor translated by these transformation,
which is desirable e.g. for unit direction vectors.

OpenGL, together with applications like OpenCV and algebraic libraries like
Eigen\footnote{eigen.org}, use homogeneous coordinates for almost all of their
work. During this project, homogeneous coordinates have been used extensively
thanks to the Eigen library, in order to have a well-defined system (Affine
transformations) to define the reference systems of interest, and in order to
compute transformations between objects. Homogeneous coordinates are converted
to native formats (e.g. COMAU robots' euler axis representation) only in the
final backend of the applications, so that data representation is unique and
coherent in the whole scope of the project.

\subsection{OpenGL reference systems} \label{opengl-reference-systems}
Opposing to a lot of 3D manipulation systems, such as COMAU's robotic arms or
CAD applications, which use different coordinate systems for different logical
aspects of their work, typically requiring at least a \emph{local} and a
\emph{global} coordinate system, OpenGL only works with a single 
coordinate system (\emph{clip coordinates}) when drawing (rendering) objects,
in which camera's frame is fixed at the origin and aligned with global axes,
with Z-axis pointing forward\footnote{This actually brings out a left-handed
coordinate system, which must be taken into account when interfacing OpenGL
with external world like it has been done in this project.},
which is then transformed into another unique coordinate system (\emph{window
coordinates}) after drawing in order to map 3D points to 2D points into the
render (camera) buffer. A third buffer (\emph{Z-depth buffer}) keeps
informations about the depth of drawn points into the visible scene and is
registered with the rendering buffer. As described in \cite{opengl-book}, in order to give the
programmer full access capabilities to the rendering engine's internals, 
multiple reference systems appear to the user only on the
OpenGL API's abstraction layer, in which different functions \emph{appear} to
work on different coordinate systems, but instead limit themselves to act
differently on the latter. Thus, with the target of emulating a physical
camera in mind, the programmer can act on the most suitable components of the
rendering process as needed, without caring about what is the right coordinate
system to transform. 

\subsection{Details about OpenGL rendering process} \label{opengl-rendering-process}
Here follows a description of the full process performed by OpenGL in order to
render a 3D mesh, from scene generation to actual mapping of 3D points into a
2D buffer. A complete and detailed description can be found in
\cite{opengl-book}, completed of actual API examples which are omitted here
for brevity.

First of all, OpenGL applications were born 

\subsection{Implementation of objects' rendering} \label{renderer3d}
\paragraph{The GLUT toolkit}
