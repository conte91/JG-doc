Robotics has always been an appealing research field for many
application fields. Since the first industrial revolution, machinery
has been built in order to automate repetitive tasks which had been
completed by humans before that moment. This process brought the
entire humanity to change its way of life, bringing the main part of
society from an agricultural, artisanal system to a modern,
industrialized one. In general, this has been pointed out as one of
the biggest turns in the human history, which changed almost every
aspect of people's life; although many changes due to the automation
processes have been seen as catastrophics by some, such as
\cite{bombarolo} and \cite{extinction}, mainly because of the
increased gap between wealthy countries and poorest ones, it is
undeniable that the increase of material wealthness which has come as a natural
consequence of all of this is a great benefit for any person.

In the first century since industrial revolution, research on
automation has focused on applying new technologic discoveries to
special-purpose machines; this has brought to automating tasks like
cloth washing, objects' manufacturing, and warehouses' management.

However, this approach has shown its drawbacks in recent times: having
a different machine for each single task, especially in case of
factory automation ones, pays a lot in terms of flexibility of the
machine itself, and this brings a much higher cost in the scenario,
which is going to be actually very common in the near future, in which
a factory wants to automate its whole structure. Also, recent
technological improvements have lowered a lot the need of
mechanics-based machinery, preferring solutions based on electronics
and complex software. This allowed a further push on automatic tasks'
completion by reducing a lot the cost of machinery (expect for
non-recurring engineering costs) due to limited material cost, which
are the most important to optimize in a world based on
mass-production. Thus, research on automation has moved to find a
solution for completing general tasks, in order to obtain, in the far
future, a robotic product which will have the ability to substitute
almost completely humans in labour.

The main research projects operating with this goal have, for obvious
reasons, focused their efforts on the study of \emph{humanoid
  robotics}, as the human body is, by an engineering point of view,
one of the most optimized, general-purpose machines in existance,
being capable of directly fulfilling extremely complex tasks, except for the
ones needing extreme precision or strength to complete. From this point of view, a
robot able to compare with the human brain in terms of decision capabilities
would make the general automation task totally completed, bringing to
the complexity of human intelligence the added value of virtually
infinite working strength and precision, due to its ad-hoc mechanical
and electronical components.

One of the main application fields in which human-like robotic devices
have always had their space is the one of objects' manipulation. In
fact, since the beginning many automatisation tasks involving
nontrivial handling of items, notably pick-and-place into
manifacturing pipelines, have been solved by usage of human-like
robotic arms, which for this reason gained their own industrial sector
since the early '70s, with the birth of companies like the Italian COMAU,
producing industrial automation solutions from the start, and the
expansion of others like the German KUKA, which moved its commercial
focus from the production of goods for manual use (wielding equipment,
communal vehicles) to the development of automated industrial robots,
which lead to the production of the first 6-DOF electric robot, named
\emph{FAMULUS}, in 1973.\footnote{http://www.kuka-robotics.com/usa/en/company/group/milestones/1973.htm}

Picking objects has thus always been a fundamental task for all of this
scenarios, and can nowadays be considered fully solved in most
cases. However, this are again all special-purpose solutions: most of
these solution are built for situations in which the pose of the
objects which will have to be manipulated is known a-priori, as they
are often coming from a preceding section of a production pipeline,
and the set of movement which the object will have to do is fixed
too. Thus, an ad-hoc solution can be studied and built, and the robot
can be thought, built and programmed to fullfill robustly its unique
task. The more general manipulation scenario, in which a robot is able
to be instructed at runtime to which object to take, how to manipulate
it, and where to find it, is far beyond the current state of the art
in robotics. Neverthless, research is growing more and more
quickly on this topic; it would, in fact, lead to virtually infinite
new opportunities for a lot of fields. One of the most important use
cases for this include
the relatively recent growth of e-commerce platforms, such at the 
most known company of this sector, \emph{Amazon.com, Inc.}, or the
online retail platform \emph{Overstock.com, Inc.}, which has become
famous as the first major e-commerce platform to accept Bitcoin as a
valid currency. Automated object manipulation is, anyway, absolutely
not limited to big companies' automation systems: another utility case
is the use of mobile, robotic platforms, equipped with robotic arms,
for domotic and household tasks, either to bring further ease in
domestic work, or to help diseased people, especially if with motion
or mental impairment. In this case, a human-like robotic arm, or a
complete humanoid robot, would be for sure the best solution, as all
the domestic scenarios are naturally designed for being interacted
with humanoids.

Research on the topic of automated object picking has, by now, started
to bring interesting and promising results; for example, universal gripping systems
are being studied for robotic arms both for industrial and domestic
scenarios (which, being so different, will hardly converge to a unique
solution); for the latter, most of the systems which have been created
take their inspiration from natural gripping systems: for example, the
solution proposed by the
robotics department of the \emph{Pisa/IIT} research centre, called
\emph{SoftHAND} and introduced in \cite{pisa-softhand}, which is shown in fig.~\ref{fig:pisa-softhand}, makes
use of a single motor to close a human-like, five-fingered hand. Each
finger, just like in a human hand, is capable of small deformations in
order to get a better grip on objects. This brings goods result in
all the scenarios in which a real human hand would suitable, which in
case of domotic areas corresponds to almost everything. Human-like
grippers are, obviously, not the only solution which is in
development: other, original systems are in study, such as Empire
Robotics' \emph{Versaball}: this gripper, which is shown in
fig.~\ref{fig:versaball}, makes use of a latex sphere filled with thin
sand, combined with a penumatic system which draws the internal air
from the sphere, thus compressing the latex: if put onto the suface of
a small, hand-pickable object, the sand will make the latex fold
perfectly the object, and when the pneumatic system is opened it will
thus close with good strength on it.

\begin{figure}[htbp]
\centering
\includegraphics[width=3in]{./Graphics/pisa-softhand}
\caption{Pisa/IIT's robotic gripping system, called \emph{SoftHAND},
  takes its design from the structure of a human hand, and is thus
  suitable for every task in which the latter would be.\label{fig:pisa-softhand}}
\end{figure}

\begin{figure}[htbp]
\centering
\includegraphics[width=3in]{./Graphics/versaball}
\caption{Empire Robotics' gripping system, called \emph{Versaball},
  diverges from the recent humanoid research and proposes a novel
  approach to the gripping problem.\label{fig:versaball}}
\end{figure}

\section{A case study for automated gripping systems: Amazon Picking
  Challenge}
