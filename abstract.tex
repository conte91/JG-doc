\chapter*{Abstract}
Robotics has always been an appealing research field for many
application fields; in particular, one of the most
important fields in which the state-of-the-art research is being
pushed forward the most is automated object grasping and manipulation,
which would have a variety of applications in heavily diversified
fields. However,  all the state-of-the-art research projects regarding
this topic mainly consist of finite, demonstration projects, and most
available development frameworks for this kind of applications are
either outdated, nonflexible, or not maintained.

This thesis project aims to deliver a new, complete framework for
automated object detection, pose estimation and grasping, with good
potentiality to be easily maintained for future use in the robotics
field. A complete detection and grasping solution has been built upon
this, which can be extended to fit diverse needs. As a reference for
the latter, the 2015 edition of the Amazon Picking Challenge has been
taken, which has been held as a part of the 2015 edition of the IEEE
International Conference on Robotics and Automation (ICRA). In this
challenge, a robot had to be able to grasp a set of known objects from
random locations within a shelf, emulating an automated warehouse
environment. Being referred to a retail warehouse, it was important
for the robot not to damage any object during its operation.

The proposed solution, and the derived framework, fit the
expectations: in particular, they contribute well to the research
fields they are put into because of the high modularity of the whole
system, which, as anticipated, has not been found in any of the
current available solutions; also, they provides the user with both a
working vision system and a working grasping strategy algorithm, which
are the main points in the solution of any grasping problem, included
the Amazon Picking Challenge one. Usage of this project in robotics
environments offers the user a single, well-defined interface for data
interchange between each of its components, based on commonly accepted
standards such as metric units for every measure; this has also been
lacking in most of the current robotics tools, mainly because of them
being composition of heavily heterogeneous libraries, each one
defining its own conventions for the specific application. This
framework can be instead used as a good starting point to build
homogeneous systems and applications, which can bring for sure
improvements to code safety, algorithms clearness, and more flexible
solutions.

The whole source code of the framework has been made available through
a permissive, open-source license, in order to facilitate knowledge
sharing in this research area and be of help for future projects.
